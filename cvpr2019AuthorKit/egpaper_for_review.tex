\documentclass[10pt,twocolumn,letterpaper]{article}

\usepackage{cvpr}
\usepackage{times}
\usepackage{epsfig}
\usepackage{graphicx}
\usepackage{amsmath}
\usepackage{amssymb}

% Include other packages here, before hyperref.

% If you comment hyperref and then uncomment it, you should delete
% egpaper.aux before re-running latex.  (Or just hit 'q' on the first latex
% run, let it finish, and you should be clear).
\usepackage[pagebackref=true,breaklinks=true,letterpaper=true,colorlinks,bookmarks=false]{hyperref}

% \cvprfinalcopy % *** Uncomment this line for the final submission

\def\cvprPaperID{****} % *** Enter the CVPR Paper ID here
\def\httilde{\mbox{\tt\raisebox{-.5ex}{\symbol{126}}}}

% Pages are numbered in submission mode, and unnumbered in camera-ready
\ifcvprfinal\pagestyle{empty}\fi
\begin{document}

%%%%%%%%% TITLE
\title{Implementation of the article \cite{Girolami2011}}

\author{Pierre Boyeau\\
{\tt\small pierre.boyeau@gmail.com}
\and
Baptiste Kerl\'eguer\\
{\tt\small baptiste.kerleguer@ens-paris-saclay.fr}
}

\maketitle
%\thispagestyle{empty}

%%%%%%%%% ABSTRACT
\begin{abstract}
On va avoir une super note!!!
\end{abstract}

%%%%%%%%% BODY TEXT
\section{Introduction}
Je pensais faire ici une explication de l'algorithme.\\
Je pense qu'il faudra faire une beamer qui est plus adapté pour la présentation mais là on est sur un document de travail. Tu peux ajouter du code. 
\section{Impl\'ementation}
Ici l'explication de notre impl\'ementation. ?? Il serait cool d'avoir de vrai données expérimentales ca te dit d'aller les chercher ici \url{https://population.un.org/wpp/Download/Standard/Population/} (Baptiste) ??
\section{Resultats}
Titre explicite non?
{\small
\bibliographystyle{ieee}
\bibliography{egbib}
}

\end{document}
