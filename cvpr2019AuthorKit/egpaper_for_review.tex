  \documentclass{beamer}
  \usepackage[utf8]{inputenc}
  \usetheme{Pittsburgh}
 
  \usepackage{algorithm}
  \usepackage{algorithmic}
  \usepackage{bm}
  

  \title{Riemann manifold Hamiltonian Monte Carlo methods}
  \author{Pierre Boyeau \and Baptiste Kerl\'eguer\\}\institute{École Normale Supérieure Paris-Saclay}
  \date{10 janvier 2018}

  \begin{document}

  \begin{frame}
  \titlepage
  \end{frame}
  
  \begin{frame}{Plan}
\tableofcontents
 \end{frame}
  
\AtBeginSection[]
{
\begin{frame}{Plan}
\tableofcontents[currentsection]
 \end{frame}
}
  \section{Context}
  
  \subsection{Hamiltonian Monte Carlo}
  \begin{frame}
  Comme décrit dans \cite{Duane1987}
  Hamiltonian : $$ H(\bm{\theta},\bm{p}) = -\mathcal{L}(\bm{\theta}) + \frac{1}{2} \log{(2\pi)^2|\bm{M}|} +\frac{1}{2} \bm{p}^T\bm{M}^{-1}\bm{p} \label{Hamiltonian}$$
  $$ \text{avec } M \text{ une matrice de masse}$$
  \'Equation d'\'evolution : 
  \begin{eqnarray}
  \frac{d\bm{\theta}}{d\tau} = \frac{\partial H}{\partial \bm{p}} \\
  \frac{d\bm{p}}{d\tau} = -\frac{\partial H}{\partial \bm{\theta}}
  \label{evolution}
  \end{eqnarray}  
  
  \end{frame}
  
  \subsection{Variété Riemannienne}
  \begin{frame}
  \cite{Girolami2011} nous décrit les changements à introduire dans le Hamiltinien
  %M\'etric : $||\bm{v}||_{\bm{M}}^2 = \bm{v}^T \bm{M} \bm{v}$ \\
  Changement de variation de masse $M$ pour prendre en compte $\bm{\theta}$ $G(\bm{\theta})$. Nous utilisons l'information de Fisher $G(\beta)=\bm{X}^T\bm{\Lambda}\bm{X}+\alpha^{-1}I$ avec $ \beta \sim \mathcal{N}(0,\alpha I) $\\
  Cependant cela induit le calcule de $G(\bm{\theta})^{-1}$ car les équations deviennent:
  \begin{eqnarray}
  \frac{d\bm{\theta}}{d\tau} = G(\bm{\theta})^{-1}\bm{p} \\
  \frac{d\bm{p}}{d\tau} = \nabla_\theta\mathcal{L}(\bm{\theta})-\frac{1}{2}\text{tr}\left(G(\bm{\theta})^{-1}\frac{\partial G(\bm{\theta})}{\partial \bm{\theta}} \right)+\frac{1}{2}\bm{p}^T(\bm{\theta})^{-1}\frac{\partial G(\bm{\theta})}{\partial \bm{\theta}}(\bm{\theta})^{-1}\bm{p}
  \label{evolution}
  \end{eqnarray} 
  
  \end{frame}
  
  \subsection{Riemann Manifold Hamiltonian Monte Carlo}
  \begin{frame}
  

\begin{algorithmic}
\REQUIRE $G(0)$ $-\mathcal{L}(\theta)$ $p^0$ $\theta^0$
%\ENSURE 
\FOR{$i = 0:N-1$}
\STATE sample $p^{i+1}$ selon $\mathcal{N}(O,G^i)$
\FOR{$j = 1 : Nb_{leapfrogs}$}
\STATE $p_{tempo}=p_\tau$
\FOR{$k = 1 : N_{pointfixe}$}
\STATE $p\left(\tau +\frac{\epsilon}{2}\right)=p\left(\tau\right)-\frac{\epsilon}{2}\nabla_\theta H\left(\theta(\tau),p_{tempo}\right)$
\ENDFOR
\STATE $p\left(\tau +\frac{\epsilon}{2}\right) =p_{tempo}$
\STATE $\theta_{tempo}=\theta(\tau)$
\FOR{$k = 1 : N_{pointfixe}$}
\STATE $\theta(\tau +\epsilon)=\theta(\tau)+\frac{\epsilon}{2}\left[G^{-1}\left(\theta(\tau)\right)+G^{-1}\left(\theta_{tempo}\right)\right]p\left(\tau+\frac{\epsilon}{2}\right)$
\ENDFOR
\STATE $\theta(\tau+\epsilon)=\theta_{tempo}$
\STATE $p(\tau+\epsilon)=p\left(\tau+\frac{\epsilon}{2}\right)-\frac{\epsilon}{2}\nabla_\theta H\left(\theta\left(\tau+\frac{\epsilon}{2}\right),p\left(\tau+\frac{\epsilon}{2}\right)\right)$
\ENDFOR
\ENDFOR
\end{algorithmic}

  \end{frame}
  
  \section{Implementation}
  \begin{frame}
  
  \end{frame}

  \begin{frame}
  \bibliographystyle{ieee}
  \bibliography{egbib}
  \end{frame}

  \end{document}